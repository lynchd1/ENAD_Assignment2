\begin{enumerate}
	\item{
	%Part a
		\begin{align*}
			F(s) &= \frac{20s^2+141s+315}{s \left(s^2+10s+21 \right)} \\
			&= \frac{20s^2+141s+315}{s(s+7)(s+3)} \\
		\end{align*}
		Perform partial fraction expansion:
		\begin{equation*}
			\frac{20s^2+141s+315}{s(s+7)(s+3)} = \frac{A}{s} + \frac{B}{s+7} + \frac{C}{s+3}
		\end{equation*}
		\begin{equation*}
			\therefore 20s^2+141s+315 = A(s+7)(s+3) + Bs(s+3) + Cs(s+7)
		\end{equation*}
		Now solve for A, B and C:
		\begin{equation*}
			s = 0 \implies 315 = A(7)(3) \implies A = 15
		\end{equation*}
		\begin{equation*}
			s = -7 \implies 308 = B(-7)(-4) \implies B = 11
		\end{equation*}
		\begin{equation*}
			s = -3 \implies 72 = A(-3)(4) \implies C = -6
		\end{equation*}
		And we arrive at $F(s)$ in partial fraction expanded form:
		\begin{equation*}
			F(s) = \frac{15}{s} + \frac{11}{s+7} - \frac{6}{s+3}
		\end{equation*}
		Now perform the inverse Laplace transform:
		\begin{align*}
			f(t) &= \mathcal{L}^{-1}[F(s)] \\
			&= \left(15 + 11e^{-7t} - 6e^{-3t} \right) u(t)
		\end{align*}
		\\
	}
	\item{
	%Part b
		\begin{align*}
			F(s) &= \frac{14s^2+56s+152}{(s+6) \left(s^2+4s+20 \right)} \\
			&= \frac{14s^2+56s+152}{(s+6)(s+2-j4)(s+2+j4)}
		\end{align*}
		Perform partial fraction expansion:
		\begin{equation*}
			\frac{14s^2+56s+152}{(s+6)(s+2-j4)(s+2+j4)} = \frac{A}{s+6} + \frac{B}{s+2-j4} + \frac{B^*}{s+2+j4}
		\end{equation*}
		\begin{equation*}
			\therefore 14s^2+56s+152 = A(s+2-j4)(s+2+j4) + B(s+6)(s+2+j4) + B^*(s+6)(s+2-j4)
		\end{equation*}
		Now solve for A and B:
		\begin{equation*}
			s = -6 \implies 320 = A(-4-j4)(-4+j4) \implies A = 10
		\end{equation*}
		\begin{equation*}
			s = -2+j4 \implies -128 = B(4+j4)(j8) \implies B = 2 + j2
		\end{equation*}
		And we arrive at $F(s)$ in partial fraction expanded form:
		\begin{align*}
			F(s) &= \frac{10}{s+6} + \frac{2+j2}{s+2-j4} + \frac{2-j2}{s+2+j4} \\
			&= \frac{10}{s+6} + \frac{2\sqrt{2} \phase{45 \degree}}{s+2-j4} + \frac{2\sqrt{2} \phase{-45 \degree}}{s+2+j4}
		\end{align*}
		Now perform the inverse Laplace transform:
		\begin{align*}
			f(t) &= \mathcal{L}^{-1}[F(s)] \\
			&= \left(10e^{-6t} + 4\sqrt{2}e^{-2t} \cos(4t + 45 \degree) \right) u(t)
		\end{align*}
		\\
	}
	\item{
	%Part c
		\begin{equation*}
			F(s) = \frac{25(s+4)^2}{s^2(s+5)^2}
		\end{equation*}
		Perform partial fraction expansion:
		\begin{equation*}
			\frac{25(s+4)^2}{s^2(s+5)^2} = \frac{A}{s^2} + \frac{B}{s} + \frac{C}{(s+5)^2} + \frac{D}{s+5}
		\end{equation*}
		\begin{equation*}
			\therefore 25(s+4)^2 = A(s+5)^2 + Bs(s+5)^2 + Cs^2 + Ds^2(s+5)
		\end{equation*}
		Now solve for A, B, C and D:
		\begin{equation*}
			s = -5 \implies 25 = C(-5)^2 \implies C = 1
		\end{equation*}
		\begin{equation*}
			s = 0 \implies 400 = A(5)^2 \implies A = 16
		\end{equation*}
		\begin{equation*}
			s = 1 \implies 625 = 36(16) + B(36) + 1 + D(6)
		\end{equation*}
		\begin{align}
			\therefore 36B + 6D &= 48 \nonumber \\
			6B + D &= 8
		\end{align}
		\begin{equation*}
			s = -1 \implies 225 = 16(16) - B(16) + 1 + D(4)
		\end{equation*}
		\begin{align}
			\therefore -16B + 4D &= -32 \nonumber \\
			-4B + D &= -8
		\end{align}
		Solving equations (1) and (2) gives $B = \frac{8}{5}$ and $D = -\frac{8}{5}$. We then arrive at $F(s)$ in partial fraction expanded form:
		\begin{equation*}
			F(s) = \frac{16}{s^2} + \frac{8}{5s} + \frac{1}{(s+5)^2} - \frac{8}{5(s+5)}
		\end{equation*}
		Now perform the inverse Laplace transform:
		\begin{align*}
			f(t) &= \mathcal{L}^{-1}[F(s)] \\
			&= \left(\frac{8}{5} + 16t + e^{-5t} \left(t - \frac{8}{5} \right) \right) u(t)
		\end{align*}
	}
	
\end{enumerate}
