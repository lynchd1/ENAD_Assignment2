\begin{enumerate}
	\item{
	% Part a
		Convert circtuit to its s-domain equivalent:
		\\ \\
		ADD CIRCUIT DIAGRAM BELOW
		\\ \\
		The circuit is a voltage divider:
		\begin{align*}
			V_o(s) &= V_i(s) * \frac{Z_R || Z_C}{Z_R || Z_C + Z_L} \\
			&= \frac{5}{s} * \frac{\frac{20s}{s(2s+10)}}{\frac{20s}{s(2s+10)}+2.5s} \\
			&= \frac{5}{s} * \frac{1}{1+0.125(2s^2+10s)} \\
			&= \frac{5}{s(0.25s^2+1.25s+1)} \\
			&= \frac{20}{s(s+4)(s+1)}
		\end{align*}
		Perform partial fraction expansion:
		\begin{align*}
			\frac{20}{s(s+4)(s+1)} &= \frac{A}{s} + \frac{B}{s+4} + \frac{C}{s+1} \\
			\therefore 20 &= A(s+4)(s+1) + Bs(s+1) + Cs(s+4)
		\end{align*}
		Now solve for A, B and C:
		\begin{equation*}
			s = 0 \implies 20 = A(4)(1) \implies A = 5
		\end{equation*}
		\begin{equation*}
			s = -4 \implies 20 = B(-4)(-3) \implies B = \frac{5}{3}
		\end{equation*}
		\begin{equation*}
			s = -1 \implies 20 = C(-1)(3) \implies C = -\frac{20}{3}
		\end{equation*}
		And we arrive at $V_o(s)$ in partial fraction expanded form:
		\begin{equation*}
			V_o(s) = \frac{5}{s} + \frac{5}{3(s+4)} - \frac{20}{3(s+1)} \ \mathrm{V}
		\end{equation*}
		\\
	}

	\item{
	%Part b
		Perform the inverse Laplace transform:
		\begin{align*}
			v_o(t) &= \mathcal{L}^{-1}\left[ V_o(s) \right] \\
			&= \left(5 + \frac{5}{3}e^{-4t} - \frac{20}{3}e^{-t} \right) u(t) \ \mathrm{V}
		\end{align*}
		\\
	}

	\item{
	%Part c
		UPDATE TO IMPROVE FLOW OF THIS PASSAGE FROM NOTES, also improve explanation for second paragraph...
		\\ \\
		Circuit elements the same, therefore $s_1, s_2$ and form of equation will remain the same. Input voltage source (forcing function) the same therefore steady state component of response will remain the same.
		\par
		One or both of the exponential terms will change, as these are the only parts of the response equation that are determined by the initial conditions.
		\\ \\
	}

\end{enumerate}
